\documentclass[xcolor=table]{beamer}

\usepackage[polish]{babel}
\usepackage[utf8]{inputenc}
\usepackage[T1]{fontenc}
\usepackage{listings}
\usepackage{lmodern}
\usepackage{textcomp}


\usetheme[language=polish]%
  {Goddard}

\newcommand{\filepath}{\texttt}
\newcommand{\command}{\texttt}
\newcommand{\email}[1]{\href{mailto:#1}{\texttt{#1}}}
\newcommand{\latexcode}{\texttt}
\newcommand{\parameter}[1]{\textlangle #1\textrangle}


\lstset{basicstyle=\ttfamily,keywordstyle=\color{goddardblue}\bfseries,commentstyle=\color{goddardblue!75}\itshape,columns=flexible}

\rowcolors{1}{goddardblue!50}{goddardblue!30}


\title{Metody rozpoznawania twarzy}
\subtitle{przeglad i porównanie}
\author{Bartłomiej Bułat\\
Tomasz Czarnik\\
Krzysztof Śmiłek\\}


\begin{document}

\begin{frame}
  \titlepage
\end{frame}


\begin{frame}
  \frametitle{Plan}
  \tableofcontents
\end{frame}


\section{Wstęp}

\begin{frame}
  \frametitle{Wstęp}

  beamer-goddard is a \LaTeX{} Beamer theme inspired by the Goddard Fedora~13 theme\footnote{\url{http://fedoraproject.org/wiki/F13_Artwork}} and the Anaconda GUI layout\footnote{\url{http://fedoraproject.org/wiki/Anaconda}}.

  \begin{block}{Comments, suggestions or bug reports ?}
    Please send a mail at: \email{melmorabity@fedoraproject.org}
  \end{block}
\end{frame}


\section{Metody rozpoznawania twarzy}

\subsection{Metody geometryczne}
\begin{frame}
  \frametitle{Metody geometryczne}

  dodać info o metodzie

\end{frame}

\begin{frame}
  \frametitle{Metody geometryczne}

  a to jest drugi slajd geometrycznych etc

\end{frame}

\subsection{Metoda analizy kolorów}
\begin{frame}
  \frametitle{Metoda analizy kolorów}

  dodać info o metodzie

\end{frame}

\subsection{Metoda sieci neuronowych}
\begin{frame}
  \frametitle{Metoda sieci neuronowych}

  dodać info o metodzie

\end{frame}

\subsection{Model aktywnego kształtu}
\begin{frame}
  \frametitle{Model aktywnego kształtu}

  dodać info o metodzie

\end{frame}


\section{Podsumowanie}

\begin{frame}[fragile]
  \frametitle{Podsumowanie}

  parę słów podsumowania tutaj
  
\end{frame}


\end{document}
